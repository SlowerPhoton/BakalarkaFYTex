\chapter{SAT}

First, we need to precisely define basic concepts we will use thorough this text.

\section{The SAT problem}
This is a famous NP-complete problem of deciding whether there exist an interpretation satisfying a given formula of propositional logic. An interpretation here is a function assigning either T (True) or F (False) to each propositional variable in the formula. We say that an interpretation $I$ satisfies a formula $\phi$ if, the evaluation of the formula, where we replace all variables with truth values according to $I$, results in T.

However, since we want to make all arguments as simple as possible, we put restrictions on what a formula can look like. We know that every propositional formula can be converted into an equivalent formula that is in conjunctive normal form (CNF). In this transformation, one can view a given formula as conjunction of clauses, which are in turn disjunctions of literals. Now, we can finally define what our instance of the problem looks like, without loss of generality.

INSTANCE: a finite set of clauses $S$

\section{The 3-SAT problem}
The instance can be further restricted to only use clauses of a given cardinality. In the 3-SAT problem we only allow clauses of cardinality equal to 3 (3-clauses).

However, we need to discuss the effect on the generality of the problem. We prove that the 3-SAT problem is as general as the SAT problem by transforming an instance of SAT into an instance of 3-SAT.

The instance of SAT is a finite set of clauses $S$. If there is a clause $C_1$, such that $|C_1| = n_1 > 3$, we transform it into $n_1-2$ new clauses. If there is a clause $C_2$, $|C_2| = n_2 < 3$, we pad this clause with new ``dummy" variables. The exact transformation manual is shown below.

Clause $C_1 = \{l_1, l_2, \ldots, l_{n_1}\}$ is replaced by new clauses $\{l_1, l_2, x_1\}$, $\{\neg x_1, l_3, x_2\}$, $\{\neg x_2, l_4, x_3\}$, \ldots, $\{\neg x_{n-4},l_{n-2},x_{n-3}\}$, $\{\neg x_{n-3}, l_{n-1}, l_n\}$, where $x_i$ are new variables not occurring elsewhere.

Clause $\{l_1\}$ is replaced by a new clause $\{l_1, y_1, y_2\}$, clause $\{l_1, l_2\}$ is likewise replaced by $\{l_1, l_2, y\}$, where $y, y_1, y_2$ are new variables not occurring elsewhere.

The last step is to show that these transformations preserve satisfiability. Let $S$ be an instance of SAT and $S_3$ the corresponding instance of 3-SAT, obtained by these transformations. If 
